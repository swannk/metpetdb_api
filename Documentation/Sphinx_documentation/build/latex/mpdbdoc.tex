% Generated by Sphinx.
\def\sphinxdocclass{report}
\documentclass[letterpaper,10pt,english]{sphinxmanual}
\usepackage[utf8]{inputenc}
\DeclareUnicodeCharacter{00A0}{\nobreakspace}
\usepackage[T1]{fontenc}
\usepackage{babel}
\usepackage{times}
\usepackage[Bjarne]{fncychap}
\usepackage{longtable}
\usepackage{sphinx}
\usepackage{multirow}


\title{mpdbdoc Documentation}
\date{February 11, 2013}
\release{1}
\author{tej}
\newcommand{\sphinxlogo}{}
\renewcommand{\releasename}{Release}
\makeindex

\makeatletter
\def\PYG@reset{\let\PYG@it=\relax \let\PYG@bf=\relax%
    \let\PYG@ul=\relax \let\PYG@tc=\relax%
    \let\PYG@bc=\relax \let\PYG@ff=\relax}
\def\PYG@tok#1{\csname PYG@tok@#1\endcsname}
\def\PYG@toks#1+{\ifx\relax#1\empty\else%
    \PYG@tok{#1}\expandafter\PYG@toks\fi}
\def\PYG@do#1{\PYG@bc{\PYG@tc{\PYG@ul{%
    \PYG@it{\PYG@bf{\PYG@ff{#1}}}}}}}
\def\PYG#1#2{\PYG@reset\PYG@toks#1+\relax+\PYG@do{#2}}

\expandafter\def\csname PYG@tok@gd\endcsname{\def\PYG@tc##1{\textcolor[rgb]{0.63,0.00,0.00}{##1}}}
\expandafter\def\csname PYG@tok@gu\endcsname{\let\PYG@bf=\textbf\def\PYG@tc##1{\textcolor[rgb]{0.50,0.00,0.50}{##1}}}
\expandafter\def\csname PYG@tok@gt\endcsname{\def\PYG@tc##1{\textcolor[rgb]{0.00,0.25,0.82}{##1}}}
\expandafter\def\csname PYG@tok@gs\endcsname{\let\PYG@bf=\textbf}
\expandafter\def\csname PYG@tok@gr\endcsname{\def\PYG@tc##1{\textcolor[rgb]{1.00,0.00,0.00}{##1}}}
\expandafter\def\csname PYG@tok@cm\endcsname{\let\PYG@it=\textit\def\PYG@tc##1{\textcolor[rgb]{0.25,0.50,0.56}{##1}}}
\expandafter\def\csname PYG@tok@vg\endcsname{\def\PYG@tc##1{\textcolor[rgb]{0.73,0.38,0.84}{##1}}}
\expandafter\def\csname PYG@tok@m\endcsname{\def\PYG@tc##1{\textcolor[rgb]{0.13,0.50,0.31}{##1}}}
\expandafter\def\csname PYG@tok@mh\endcsname{\def\PYG@tc##1{\textcolor[rgb]{0.13,0.50,0.31}{##1}}}
\expandafter\def\csname PYG@tok@cs\endcsname{\def\PYG@tc##1{\textcolor[rgb]{0.25,0.50,0.56}{##1}}\def\PYG@bc##1{\setlength{\fboxsep}{0pt}\colorbox[rgb]{1.00,0.94,0.94}{\strut ##1}}}
\expandafter\def\csname PYG@tok@ge\endcsname{\let\PYG@it=\textit}
\expandafter\def\csname PYG@tok@vc\endcsname{\def\PYG@tc##1{\textcolor[rgb]{0.73,0.38,0.84}{##1}}}
\expandafter\def\csname PYG@tok@il\endcsname{\def\PYG@tc##1{\textcolor[rgb]{0.13,0.50,0.31}{##1}}}
\expandafter\def\csname PYG@tok@go\endcsname{\def\PYG@tc##1{\textcolor[rgb]{0.19,0.19,0.19}{##1}}}
\expandafter\def\csname PYG@tok@cp\endcsname{\def\PYG@tc##1{\textcolor[rgb]{0.00,0.44,0.13}{##1}}}
\expandafter\def\csname PYG@tok@gi\endcsname{\def\PYG@tc##1{\textcolor[rgb]{0.00,0.63,0.00}{##1}}}
\expandafter\def\csname PYG@tok@gh\endcsname{\let\PYG@bf=\textbf\def\PYG@tc##1{\textcolor[rgb]{0.00,0.00,0.50}{##1}}}
\expandafter\def\csname PYG@tok@ni\endcsname{\let\PYG@bf=\textbf\def\PYG@tc##1{\textcolor[rgb]{0.84,0.33,0.22}{##1}}}
\expandafter\def\csname PYG@tok@nl\endcsname{\let\PYG@bf=\textbf\def\PYG@tc##1{\textcolor[rgb]{0.00,0.13,0.44}{##1}}}
\expandafter\def\csname PYG@tok@nn\endcsname{\let\PYG@bf=\textbf\def\PYG@tc##1{\textcolor[rgb]{0.05,0.52,0.71}{##1}}}
\expandafter\def\csname PYG@tok@no\endcsname{\def\PYG@tc##1{\textcolor[rgb]{0.38,0.68,0.84}{##1}}}
\expandafter\def\csname PYG@tok@na\endcsname{\def\PYG@tc##1{\textcolor[rgb]{0.25,0.44,0.63}{##1}}}
\expandafter\def\csname PYG@tok@nb\endcsname{\def\PYG@tc##1{\textcolor[rgb]{0.00,0.44,0.13}{##1}}}
\expandafter\def\csname PYG@tok@nc\endcsname{\let\PYG@bf=\textbf\def\PYG@tc##1{\textcolor[rgb]{0.05,0.52,0.71}{##1}}}
\expandafter\def\csname PYG@tok@nd\endcsname{\let\PYG@bf=\textbf\def\PYG@tc##1{\textcolor[rgb]{0.33,0.33,0.33}{##1}}}
\expandafter\def\csname PYG@tok@ne\endcsname{\def\PYG@tc##1{\textcolor[rgb]{0.00,0.44,0.13}{##1}}}
\expandafter\def\csname PYG@tok@nf\endcsname{\def\PYG@tc##1{\textcolor[rgb]{0.02,0.16,0.49}{##1}}}
\expandafter\def\csname PYG@tok@si\endcsname{\let\PYG@it=\textit\def\PYG@tc##1{\textcolor[rgb]{0.44,0.63,0.82}{##1}}}
\expandafter\def\csname PYG@tok@s2\endcsname{\def\PYG@tc##1{\textcolor[rgb]{0.25,0.44,0.63}{##1}}}
\expandafter\def\csname PYG@tok@vi\endcsname{\def\PYG@tc##1{\textcolor[rgb]{0.73,0.38,0.84}{##1}}}
\expandafter\def\csname PYG@tok@nt\endcsname{\let\PYG@bf=\textbf\def\PYG@tc##1{\textcolor[rgb]{0.02,0.16,0.45}{##1}}}
\expandafter\def\csname PYG@tok@nv\endcsname{\def\PYG@tc##1{\textcolor[rgb]{0.73,0.38,0.84}{##1}}}
\expandafter\def\csname PYG@tok@s1\endcsname{\def\PYG@tc##1{\textcolor[rgb]{0.25,0.44,0.63}{##1}}}
\expandafter\def\csname PYG@tok@gp\endcsname{\let\PYG@bf=\textbf\def\PYG@tc##1{\textcolor[rgb]{0.78,0.36,0.04}{##1}}}
\expandafter\def\csname PYG@tok@sh\endcsname{\def\PYG@tc##1{\textcolor[rgb]{0.25,0.44,0.63}{##1}}}
\expandafter\def\csname PYG@tok@ow\endcsname{\let\PYG@bf=\textbf\def\PYG@tc##1{\textcolor[rgb]{0.00,0.44,0.13}{##1}}}
\expandafter\def\csname PYG@tok@sx\endcsname{\def\PYG@tc##1{\textcolor[rgb]{0.78,0.36,0.04}{##1}}}
\expandafter\def\csname PYG@tok@bp\endcsname{\def\PYG@tc##1{\textcolor[rgb]{0.00,0.44,0.13}{##1}}}
\expandafter\def\csname PYG@tok@c1\endcsname{\let\PYG@it=\textit\def\PYG@tc##1{\textcolor[rgb]{0.25,0.50,0.56}{##1}}}
\expandafter\def\csname PYG@tok@kc\endcsname{\let\PYG@bf=\textbf\def\PYG@tc##1{\textcolor[rgb]{0.00,0.44,0.13}{##1}}}
\expandafter\def\csname PYG@tok@c\endcsname{\let\PYG@it=\textit\def\PYG@tc##1{\textcolor[rgb]{0.25,0.50,0.56}{##1}}}
\expandafter\def\csname PYG@tok@mf\endcsname{\def\PYG@tc##1{\textcolor[rgb]{0.13,0.50,0.31}{##1}}}
\expandafter\def\csname PYG@tok@err\endcsname{\def\PYG@bc##1{\setlength{\fboxsep}{0pt}\fcolorbox[rgb]{1.00,0.00,0.00}{1,1,1}{\strut ##1}}}
\expandafter\def\csname PYG@tok@kd\endcsname{\let\PYG@bf=\textbf\def\PYG@tc##1{\textcolor[rgb]{0.00,0.44,0.13}{##1}}}
\expandafter\def\csname PYG@tok@ss\endcsname{\def\PYG@tc##1{\textcolor[rgb]{0.32,0.47,0.09}{##1}}}
\expandafter\def\csname PYG@tok@sr\endcsname{\def\PYG@tc##1{\textcolor[rgb]{0.14,0.33,0.53}{##1}}}
\expandafter\def\csname PYG@tok@mo\endcsname{\def\PYG@tc##1{\textcolor[rgb]{0.13,0.50,0.31}{##1}}}
\expandafter\def\csname PYG@tok@mi\endcsname{\def\PYG@tc##1{\textcolor[rgb]{0.13,0.50,0.31}{##1}}}
\expandafter\def\csname PYG@tok@kn\endcsname{\let\PYG@bf=\textbf\def\PYG@tc##1{\textcolor[rgb]{0.00,0.44,0.13}{##1}}}
\expandafter\def\csname PYG@tok@o\endcsname{\def\PYG@tc##1{\textcolor[rgb]{0.40,0.40,0.40}{##1}}}
\expandafter\def\csname PYG@tok@kr\endcsname{\let\PYG@bf=\textbf\def\PYG@tc##1{\textcolor[rgb]{0.00,0.44,0.13}{##1}}}
\expandafter\def\csname PYG@tok@s\endcsname{\def\PYG@tc##1{\textcolor[rgb]{0.25,0.44,0.63}{##1}}}
\expandafter\def\csname PYG@tok@kp\endcsname{\def\PYG@tc##1{\textcolor[rgb]{0.00,0.44,0.13}{##1}}}
\expandafter\def\csname PYG@tok@w\endcsname{\def\PYG@tc##1{\textcolor[rgb]{0.73,0.73,0.73}{##1}}}
\expandafter\def\csname PYG@tok@kt\endcsname{\def\PYG@tc##1{\textcolor[rgb]{0.56,0.13,0.00}{##1}}}
\expandafter\def\csname PYG@tok@sc\endcsname{\def\PYG@tc##1{\textcolor[rgb]{0.25,0.44,0.63}{##1}}}
\expandafter\def\csname PYG@tok@sb\endcsname{\def\PYG@tc##1{\textcolor[rgb]{0.25,0.44,0.63}{##1}}}
\expandafter\def\csname PYG@tok@k\endcsname{\let\PYG@bf=\textbf\def\PYG@tc##1{\textcolor[rgb]{0.00,0.44,0.13}{##1}}}
\expandafter\def\csname PYG@tok@se\endcsname{\let\PYG@bf=\textbf\def\PYG@tc##1{\textcolor[rgb]{0.25,0.44,0.63}{##1}}}
\expandafter\def\csname PYG@tok@sd\endcsname{\let\PYG@it=\textit\def\PYG@tc##1{\textcolor[rgb]{0.25,0.44,0.63}{##1}}}

\def\PYGZbs{\char`\\}
\def\PYGZus{\char`\_}
\def\PYGZob{\char`\{}
\def\PYGZcb{\char`\}}
\def\PYGZca{\char`\^}
\def\PYGZam{\char`\&}
\def\PYGZlt{\char`\<}
\def\PYGZgt{\char`\>}
\def\PYGZsh{\char`\#}
\def\PYGZpc{\char`\%}
\def\PYGZdl{\char`\$}
\def\PYGZti{\char`\~}
% for compatibility with earlier versions
\def\PYGZat{@}
\def\PYGZlb{[}
\def\PYGZrb{]}
\makeatother

\begin{document}

\maketitle
\tableofcontents
\phantomsection\label{index::doc}


This is a documentation of the \href{http://metpetdb.rpi.edu/metpetweb/\#home}{MetPetDB} system.

Contents:


\chapter{Introduction to MetPetDB}
\label{Introduction:introduction}\label{Introduction:introduction-to-metpetdb}\label{Introduction:welcome-to-metpetdb-s-documentation}\label{Introduction::doc}
MetPetDB is a database for metamorphic petrology that is being designed and built by a global
community of metamorphic petrologists in collaboration with computer scientists at Rensselaer
Polytechnic Institute as part of the National Cyberinfrastructure Initiative and supported by the
National Science Foundation.

The details of the project can be found \href{http://metpetdb.rpi.edu/metpetweb/\#home}{here}.

This document explains the various tables and their attributes in the database schema.

The schema can be referenced \href{https://github.com/metpetdb/metpetdb/tree/master/mpdb-server/schema}{here}. This may be reviesed slightly. It would be updated shortly.

The Entity Relationship diagram can be found here (link to github after upload).


\chapter{Table Description}
\label{Table_Description:table-description}\label{Table_Description::doc}\label{Table_Description:id1}

\section{List of Tables}
\label{Table_Description:list-of-tables}\label{Table_Description:installing-docdir}

\subsection{Samples}
\label{Table_Description:samples}
Stores information regarding the sample through a unique sample id, rock type id , latitude and
longitude. More information regarding the sample object and its attributes can be found at \href{http://wiki.cs.rpi.edu/trac/metpetdb/wiki/SampleObject}{http://wiki.cs.rpi.edu/trac/metpetdb/wiki/SampleObject}

\begin{tabulary}{\linewidth}{|L|L|L|}
\hline
\textbf{
Attributes
} & \textbf{
Type
} & \textbf{
Explanation
}\\\hline

sample\_id
 & 
Required(PK)
 & 
Unique id for the sample for that particular owner
\\\hline

version
 & 
Required
 & 
It is the number of times that specific tuple has been modified(used
for concurrency issues).You read a tuple with its version info. You
read a tuple with its version info. When trying to save it, you check
whether the version has not changed since you read it. If not, you
can write the new update. Otherwise, the update can fail. More
information \href{http://wiki.cs.rpi.edu/trac/metpetdb/wiki/Versioning}{here}
\\\hline

sesar\_number
 & 
Optional
 & 
Unique identifiers created by SESAR. Sample will have a sesar\_number if
it is registered here (System for Earth SAmple Registration -
\href{http://www.geosamples.org/}{http://www.geosamples.org/}
\\\hline

public\_data
 & 
Required
 & 
Y/N depending on if the sample is public /
private/project
\\\hline

collection\_date
 & 
Optional
 & 
Must be the date when sample was collected in MM-DD-YYYY, YYYY-MM-DD
or a shortened version
\\\hline

date\_precision
 & 
Optional
 & 
Date could be a specific day and time (0), a specific day (1), a month
and year (31), or a year (365). Currently stored values
are 0 1 31 365
\\\hline

number
 & 
Required
 & 
Alternate sample number. Generally the number given by the person.
\\\hline

rock\_type\_id
 & 
Required(FK)
 & 
References the rock\_type\_id from the table rock\_type. Every sample in the
database must be one of the rock types as described \href{http://wiki.cs.rpi.edu/trac/metpetdb/wiki/RockType}{here}
\\\hline

user\_id
 & 
Required(FK)
 & 
References the users table  which stores all the user information.
\\\hline

location\_error
 & 
Optional
 & 
Error in meters from the actual location of the samples.
(latitude/longitude errors)
\\\hline

country
 & 
Optional
 & 
Country where the sample was collected
\\\hline

description
 & 
Optional
 & 
Text data about the sample.
\\\hline

collector
 & 
Optional
 & 
Name of the person who collected the sample in the Lastname, Firstname format.
\\\hline

metamor
 &  & \\\hline

phic\_region\_id
 & 
Optional
 & 
Should have been a FK as the metamorphic\_regions table stores the
metamorphic\_region\_id as a required field.
\\\hline

collector\_id
 & 
FK
 & 
References the users for the user\_id. Collector\_id same as user\_id?
\\\hline
\end{tabulary}



\subsection{subsamples}
\label{Table_Description:subsamples}
A thin subsection of the sample is called a subsample. subsamples table stores the details of the sub samples for a particular sample. Sub sample description can be found \href{http://wiki.cs.rpi.edu/trac/metpetdb/wiki/Subsample}{here}

\begin{tabulary}{\linewidth}{|L|L|L|}
\hline
\textbf{
Attributes
} & \textbf{
Type
} & \textbf{
Explanation
}\\\hline

subsample\_id
 & 
Required (PK)
 & 
Unique id for the subsample
\\\hline

version
 & 
Required
 & 
It is the number of times that specific tuple has been modified(used
for concurrency issues).You read a tuple with its version info.
When trying to save it, you check whether the version has not changed
since you read it. If not, you can write the new update.
Otherwise, the update can fail. More
information \href{http://wiki.cs.rpi.edu/trac/metpetdb/wiki/Versioning}{here}
\\\hline

public\_data
 & 
Required
 & 
Y/N depending on if the subsample is public / private/published
\\\hline

sample\_id
 & 
Required (FK)
 & 
References samples table to get the corresponding sample\_id.
\\\hline

user\_id
 & 
Required
 & 
References the users table  which stores all information regarding
the users
\\\hline

grid\_id
 & 
Optional
 & 
This information is stored if there is an image map associated
with the subsample. This information is shown depending on if the
subsample is private/public.
\\\hline

name
 & 
Required
 & 
There are 4 types of subsamples. This stores the name of the subsample.
\\\hline

subsample\_type\_id
 & 
Required
 & 
Subsample type id is stored here.
\\\hline
\end{tabulary}



\subsection{chemical\_analyses}
\label{Table_Description:chemical-analyses}
Stores the chemical information for a particular subsample and associated with one sample. Hence, references subsamples and samples table through Foreign Key. The attributes can be seen in detail \href{http://wiki.cs.rpi.edu/trac/metpetdb/wiki/ChemicalAnalysisObject}{here}

\begin{tabulary}{\linewidth}{|L|L|L|}
\hline
\textbf{
Attributes
} & \textbf{
Type
} & \textbf{
Explanation
}\\\hline

chemical\_analysis\_id
 & 
Required (PK)
 & 
Unique id for the chemical analysis of a subsample.
\\\hline

version
 & 
Required
 & 
It is the number of times that specific tuple has been modified
(used for concurrency issues). You read a tuple with its version info.
When trying to save it, you check whether the version has not changed
since you read it. If not, you can write the new update. Otherwise,
the update can fail. More information \href{http://wiki.cs.rpi.edu/trac/metpetdb/wiki/Versioning}{here}
\\\hline

spot\_id
 & 
Required
 & 
``Spot\_id'' is the unique identifier for a chemical analysis.
It corresponds to the point number in an electron microprobe
analytical session. t is user specified and required to be unique
for a subsample
\\\hline

subsample\_id
 & 
Required (FK)
 & 
References the subsamples table for the subsample it is
associated with.
\\\hline

public\_data
 & 
Required
 & 
Y/N depending on if the chemical\_analysis is
public / private/project
\\\hline

reference\_x
 & 
Optional
 & 
The x coordinate location of the analysis on the reference image,
measured in percent of total image width (in original orientaion).
The origin is assumed to be at bottom left of image.
\\\hline

reference\_y
 & 
Optional
 & 
The y coordinate location of the analysis on the reference image,
measured in percent of total image height (in original orientaion).
The origin is assumed to be at bottom left of image.
\\\hline

stage\_x
 & 
Optional
 & 
The stage X-coordinate recorded by a microscope or microprobe.
These are in microns.
\\\hline

stage\_y
 & 
Optional
 & 
The stage Y-coordinate recorded by a microscope or microprobe.
These are in microns.
\\\hline

image\_id
 & 
Optional (FK)
 & 
The id of the image on which the analysis location is referenced.
\\\hline

analysis\_method
 & 
Optional
 & 
Method of analysis from the accepted \href{http://wiki.cs.rpi.edu/trac/metpetdb/wiki/Methods}{method code}
\\\hline

where\_done
 & 
Optional
 & 
The analytical facility where analysis was performed
\\\hline

analyst
 & 
Optional
 & 
Name of the person who analysed the sample. Generally as Lastname,
Firstname.
\\\hline

analysis\_date
 & 
Optional
 & 
Date on which the analysis was performed.
\\\hline

date\_precision
 & 
Optional
 & 
Date could be a specific day and time (0), a specific day (1),
a month  and year (31), or a year (365). Currently stored values are:
0, 1, 31, 365
\\\hline

reference\_id
 & 
Optional (FK)
 & 
reference\_id from reference table related to this chemical analysis.
\\\hline

description
 & 
Optional
 & 
Text description of the analytical strategy.
\\\hline

mineral\_id
 & 
Optional (FK)
 & 
The id for the minerals listed for that particular sample for
which the chemical analysis is being performed.
References the minerals table.
\\\hline

user\_id
 & 
Required (FK)
 & 
References the users table to obtain the owner of the data.
\\\hline

large\_rock
 & 
Required
 & 
Y/N. ‘Y’ for bulk\_rock analysis, ‘N’ for spot analysis.
\\\hline

total
 & 
Optional
 & 
The total weight percent of measured elements/species for this point.
This indicates the completeness of the analysis to a user.
\\\hline
\end{tabulary}



\subsection{users}
\label{Table_Description:users}
Stores the information of the users through a unique user\_id (primary key) and  email (this may be changed later on) and references the roles table for role\_id as users can have many roles.
Refer \href{http://wiki.cs.rpi.edu/trac/metpetdb/wiki/UserTypes}{http://wiki.cs.rpi.edu/trac/metpetdb/wiki/UserTypes} for information on roles/user types.

\begin{tabulary}{\linewidth}{|L|L|L|}
\hline
\textbf{
Attributes
} & \textbf{
Type
} & \textbf{
Explanation
}\\\hline

user\_id
 & 
Required (PK)
 & 
Unique id for each user.
\\\hline

version
 & 
Required
 & 
It is the number of times that specific tuple has been modified
(used for concurrency issues). You read a tuple with its version info.
When trying to save it, you check whether the version has not changed
since you read it. If not, you can write the new update. Otherwise,
the update can fail. More information \href{http://wiki.cs.rpi.edu/trac/metpetdb/wiki/Versioning}{here}
\\\hline

name
 & 
Required
 & 
Name of the user
\\\hline

email
 & 
Required (unique)
 & 
Email id which is used as a login
\\\hline

password
 & 
Required
 & 
Password chosen by users
\\\hline

address
 & 
Optional
 & 
Optional information
\\\hline

city
 & 
Optional
 & 
Optional information
\\\hline

province
 & 
Optional
 & 
Optional information
\\\hline

country
 & 
Optional
 & 
Optional information
\\\hline

postal\_code
 & 
Optional
 & 
Optional information
\\\hline

institution
 & 
Optional
 & 
Optional information
\\\hline

reference\_email
 & 
Optional
 & 
Optional information
\\\hline

confirmation\_code
 & 
Optional
 & 
Not sure, may be no confirmation code for users who are just
regular users (i.e. no upload privileges) or it could  optional
due to legacy reasons.
\\\hline

enabled
 & 
Required
 & 
Single character for enabled/disabled status.
\\\hline

role\_id
 & 
Required (FK)
 & 
References the roles table for the role\_id of the user.
\\\hline
\end{tabulary}



\subsection{roles}
\label{Table_Description:roles}
Stores the role information for the user through the following attributes. This table may be subsumed by Drupal as a new user must be created for multiple roles for a user. The various roles of a user can be found \href{http://wiki.cs.rpi.edu/trac/metpetdb/wiki/UserTypes}{here}

\begin{tabulary}{\linewidth}{|L|L|L|}
\hline
\textbf{
Attributes
} & \textbf{
Type
} & \textbf{
Explanation
}\\\hline

role\_id
 & 
Required (PK)
 & 
Unique id for a particular role
\\\hline

rank
 & 
Required (unique)
 & 
Rank of the role ; for eg: 0 -\textgreater{} Member, 1-\textgreater{} contributor etc.
\\\hline

role\_name
 & 
Required
 & 
Name of the role. Eg: Member, contributor, fellow etc.
\\\hline
\end{tabulary}



\subsection{sample\_aliases}
\label{Table_Description:sample-aliases}
A sample can have multiple aliases and it is stored in the sample\_aliases table with  sample\_id and alias name together being unique. (i.e same sample is referred by different numbers sometimes)
(even in the presence of a unique constraint it is possible to store duplicate rows that contain a null value in at least one of the constrained columns)

\begin{tabulary}{\linewidth}{|L|L|L|}
\hline
\textbf{
Attributes
} & \textbf{
Type
} & \textbf{
Explanation
}\\\hline

sample\_alias\_id
 & 
Required(PK)
 & 
Alias id is the alternate sample number
\\\hline

sample\_id
 & 
Required, FK(unique)
 & 
The actual sample\_id whose alias is being recorded.
\\\hline

alias
 & 
Required (unique)
 & 
Name of the alias.
\\\hline
\end{tabulary}



\subsection{sample\_comments}
\label{Table_Description:sample-comments}
This table stores the comments associated with a sample by referencing the sample\_id from the samples table. It also shows the owner of the comment by referencing the user\_id from the users table.

\begin{tabulary}{\linewidth}{|L|L|L|}
\hline
\textbf{
Attributes
} & \textbf{
Type
} & \textbf{
Explanation
}\\\hline

comment\_id
 & 
Required (PK)
 & 
Id of the comment (integer)
\\\hline

sample\_id
 & 
Required (FK)
 & 
Referenced from samples table
\\\hline

user\_id
 & 
Required (FK)
 & 
Owner of the comment referenced from the users table
\\\hline

comment\_text
 & 
Required
 & 
Comment text
\\\hline

date\_added
 & 
Optional
 & 
Date on which the comment was added.
\\\hline
\end{tabulary}



\subsection{rock\_type}
\label{Table_Description:rock-type}
This table stores the information regarding the rock type through the rock\_type\_id which is a primary key and the name of the type of rock through the attribute rock\_type.
The trac wiki \href{http://wiki.cs.rpi.edu/trac/metpetdb/wiki/RockType}{http://wiki.cs.rpi.edu/trac/metpetdb/wiki/RockType} lists the various rock types.

\begin{tabulary}{\linewidth}{|L|L|L|}
\hline
\textbf{
Attributes
} & \textbf{
Type
} & \textbf{
Explanation
}\\\hline

rock\_type\_id
 & 
Required(PK)
 & 
Unique id of the type of rock from the above list
\\\hline

rock\_type
 & 
Required (unique)
 & 
Name of the rock type from the above list
\\\hline
\end{tabulary}



\subsection{subsample\_type}
\label{Table_Description:subsample-type}
This table stores the information regarding the type of subsample through subsample\_type\_id and subsample\_type(name of subsample). The list of subsample types can be found in the subsample wiki
\href{http://wiki.cs.rpi.edu/trac/metpetdb/wiki/Subsample}{http://wiki.cs.rpi.edu/trac/metpetdb/wiki/Subsample}

\begin{tabulary}{\linewidth}{|L|L|L|}
\hline
\textbf{
Attributes
} & \textbf{
Type
} & \textbf{
Explanation
}\\\hline

subsample\_type\_id
 & 
Required (PK)
 & 
Unique id of the subsample type
\\\hline

subsample\_type
 & 
Required (unique)
 & 
Name of the subsample type
\\\hline
\end{tabulary}



\subsection{chemical\_analysis\_elements}
\label{Table_Description:chemical-analysis-elements}
This table stores the details of the elements in the chemical analysis of a particular sample. It references the chemical\_analyses and elements table to obtain the chemical\_analysis\_id and the element\_id. It also stores the precision of the measured value through the “precision” attribute.

(it is a correlation of chemical\_analyses of a sample/subsample with the elements)
Details of chemical\_analyses can be found \href{http://wiki.cs.rpi.edu/trac/metpetdb/wiki/ChemicalAnalysisObject}{here}

List of elements can be found \href{http://wiki.cs.rpi.edu/trac/metpetdb/wiki/Element}{here}

\begin{tabulary}{\linewidth}{|L|L|L|}
\hline
\textbf{
Attributes
} & \textbf{
Type
} & \textbf{
Explanation
}\\\hline

chemical\_analysis\_id
 & 
Required (PK, FK)
 & 
The unique id by referencing the chemical\_analyses table
\\\hline

element\_id
 & 
Required (PK, FK)
 & 
The unique id for the element by referencing the elements table.
\\\hline

amount
 & 
Required
 & 
Amount of the element present
\\\hline

precision
 & 
Optional
 & 
Indicates the precision of the value measured
\\\hline

precision\_type
 & 
Optional
 & 
Type of precision in terms of absolute or relative. (‘ABS’, ‘REL’)
\\\hline

measurement\_unit
 & 
Optional
 & 
Unit used to measure the amount of element for a particular
chemical\_analyses\_id. (4 characters)
\\\hline

min\_amount
 & 
Optional
 & 
When ``precision'' is specified AND ``precision\_type''= REL, then
``min\_amount''= ``amount'' - ``precision'' X ``amount'' and
``max\_amount'' = ``amount'' + ``precision'' X ``amount''.
\\\hline

max\_amount
 & 
Optional
 & 
when ``precision'' is specified AND ``precision\_type''= ABS, then
``min\_amount''= ``amount'' - ``precision''  and
``max\_amount'' = ``amount'' + ``precision''
\\\hline
\end{tabulary}



\subsection{minerals}
\label{Table_Description:minerals}
This table stores the mineral\_id and name of the mineral. It also stores the real\_mineral\_id which is the default for alternative minerals.
The trac wiki lists the minerals \href{http://wiki.cs.rpi.edu/trac/metpetdb/wiki/Mineral}{http://wiki.cs.rpi.edu/trac/metpetdb/wiki/Mineral}

\begin{tabulary}{\linewidth}{|L|L|L|}
\hline
\textbf{
Attributes
} & \textbf{
Type
} & \textbf{
Explanation
}\\\hline

mineral\_id
 & 
Required (PK)
 & 
Unique id for the mineral
\\\hline

real\_mineral\_id
 & 
Required (FK)
 & 
for alternative minerals this is the default id, else it is
mineral\_id. References minerals table itself.
\\\hline

name
 & 
Required (unique)
 & 
Name of the mineral corresponding to
mineral\_id
\\\hline
\end{tabulary}



\subsection{mineral\_relationships}
\label{Table_Description:mineral-relationships}
This table stores the relationship between minerals. Parent mineral, child mineral etc. The mineral ids are stored by referencing the “minerals” table.

\begin{tabulary}{\linewidth}{|L|L|L|}
\hline
\textbf{
Attributes
} & \textbf{
Type
} & \textbf{
Explanation
}\\\hline

parent\_mineral\_id
 & 
Required (PK, FK)
 & 
Id of the parent mineral by referencing the minerals Table
\\\hline

child\_mineral\_id
 & 
Required(PK, FK)
 & 
Id of the child mineral by referencing the minerals table.
\\\hline
\end{tabulary}



\subsection{element\_mineral\_types}
\label{Table_Description:element-mineral-types}
Chemical analyses of the minerals is at the core of quantitative metamorphic geochemistry and minerals consist of elements.
This table correlates elements and minerals.
List of elements -  \href{http://wiki.cs.rpi.edu/trac/metpetdb/wiki/Element}{http://wiki.cs.rpi.edu/trac/metpetdb/wiki/Element}

list of Minerals - \href{http://wiki.cs.rpi.edu/trac/metpetdb/wiki/Mineral}{http://wiki.cs.rpi.edu/trac/metpetdb/wiki/Mineral}

\begin{tabulary}{\linewidth}{|L|L|L|}
\hline
\textbf{
Attributes
} & \textbf{
Type
} & \textbf{
Explanation
}\\\hline

element\_id
 & 
Required (PK, FK)
 & 
References elements table
\\\hline

mineral\_type\_id
 & 
Required (PK, FK)
 & 
References minerals table
\\\hline
\end{tabulary}



\subsection{oxide\_mineral\_types}
\label{Table_Description:oxide-mineral-types}
This table stores the oxides for a particular mineral.  Hence both the mineral\_type\_id and oxide\_id are Primary keys. Hence, neither of them can be NULL and they are together unique.

The list of oxides \href{http://wiki.cs.rpi.edu/trac/metpetdb/wiki/Oxide}{http://wiki.cs.rpi.edu/trac/metpetdb/wiki/Oxide}

The list of Minerals \href{http://wiki.cs.rpi.edu/trac/metpetdb/wiki/Mineral}{http://wiki.cs.rpi.edu/trac/metpetdb/wiki/Mineral}

\begin{tabulary}{\linewidth}{|L|L|L|}
\hline
\textbf{
Attributes
} & \textbf{
Type
} & \textbf{
Explanation
}\\\hline

oxide\_id
 & 
Required (PK, FK)
 & 
References oxides table
\\\hline

mineral\_type\_id
 & 
Required (PK, FK)
 & 
References minerals table
\\\hline
\end{tabulary}



\subsection{elements}
\label{Table_Description:elements}
This table stores the information of the elements in the chemical data of the sample.
The list of elements can be found here in the trac wiki \href{http://wiki.cs.rpi.edu/trac/metpetdb/wiki/Element}{http://wiki.cs.rpi.edu/trac/metpetdb/wiki/Element}

\begin{tabulary}{\linewidth}{|L|L|L|}
\hline
\textbf{
Attributes
} & \textbf{
Type
} & \textbf{
Explanation
}\\\hline

element\_id
 & 
Required (PK)
 & 
Unique id for the element in the list of elements
\\\hline

name
 & 
Required (unique)
 & 
Name of the element – upto 100 characters allowed
\\\hline

alternate\_name
 & 
Required
 & 
Alternate name – upto 100 characters
\\\hline

symbol
 & 
Required (unique)
 & 
Symbol upto 4 character length
\\\hline

atomic\_number
 & 
Required
 & 
Is an integer
\\\hline

weight
 & 
Optional
 & 
Atomic weight
\\\hline
\end{tabulary}



\subsection{chemical\_analysis\_oxides}
\label{Table_Description:chemical-analysis-oxides}
This table stores the oxide information for a chemical\_analysis\_id. i.e both oxide\_id and chemical\_analysis\_id are together unique (Primary Keys). It references the tables chemical\_analyses and oxides. (correlation of chemical\_analyses of a sample/subsample with the oxide)

List of oxides can be found \href{http://wiki.cs.rpi.edu/trac/metpetdb/wiki/Oxide}{here}

\begin{tabulary}{\linewidth}{|L|L|L|}
\hline
\textbf{
Attributes
} & \textbf{
Type
} & \textbf{
Explanation
}\\\hline

chemical\_analysis\_id
 & 
Required (PK, FK)
 & 
The unique id by referencing the chemical\_analyses table
\\\hline

oxide\_id
 & 
Required (PK, FK)
 & 
The unique id for the oxide by referencing the oxide table.
\\\hline

amount
 & 
Required
 & 
Amount of the oxide present
\\\hline

precision
 & 
Optional
 & 
Indicates the precision of the value measured
\\\hline

precision\_type
 & 
Optional
 & 
Type of precision in terms of absolute or relative. (‘ABS’, ‘REL’)
\\\hline

measurement\_unit
 & 
Optional
 & 
Unit used to measure the amount of oxidet for a particular
chemical\_analyses\_id. (4 characters)
\\\hline

min\_amount
 & 
Optional
 & 
When ``precision'' is specified AND ``precision\_type''= REL, then
``min\_amount''= ``amount'' - ``precision'' X ``amount'' and
``max\_amount'' = ``amount'' + ``precision'' X ``amount''.
\\\hline

max\_amount
 & 
Optional
 & 
when ``precision'' is specified AND ``precision\_type''= ABS, then
``min\_amount''= ``amount'' - ``precision''  and
``max\_amount'' = ``amount'' + ``precision''
\\\hline
\end{tabulary}



\subsection{oxides}
\label{Table_Description:oxides}
This table stores the oxide information (oxide\_id) for each element\_id, which is referenced from the elements table.
The list of oxides can be found in the trac wiki \emph{\textless{}http://wiki.cs.rpi.edu/trac/metpetdb/wiki/Oxide\textgreater{}}

\begin{tabulary}{\linewidth}{|L|L|L|}
\hline
\textbf{
Attributes
} & \textbf{
Type
} & \textbf{
Explanation
}\\\hline

oxide\_id
 & 
Required (PK)
 & 
Unique id of the oxide obtained by referencing the oxides table
\\\hline

element\_id
 & 
Required (FK)
 & 
The element\_id from the list of elements corresponding to the oxide
\\\hline

oxidation\_state
 & 
Optional
 & 
Is an integer representing the oxidation state of the atom of the oxide
\\\hline

species
 & 
Unique
 & 
Group of elements are a part of a species. Species name, list:
Ce2O3, Al2O3, etc..
\\\hline

weight
 & 
Optional
 & 
Molecular weight
\\\hline

cations\_per\_oxide
 & 
Optional
 & 
Number of cations per oxide stored as an integer.
\\\hline

conversion\_factor
 & 
Required
 & 
This is the factor used to convert from element atomic weight
(``weight'' in the element table) to a cation unit oxide weight
(``weight'' in the oxide table divided by ``cations\_per\_oxide'' in
the oxide table)
\\\hline
\end{tabulary}



\subsection{images}
\label{Table_Description:images}
This table stores the image details along with rock descriptions and chemical\_analyses. Every image is associated with one or more samples and one or more sub samples. Hence, it references the tables samples for sample\_id and subsamples for subsample\_id.
The details about the image object can be obtained from the following wiki
\href{http://wiki.cs.rpi.edu/trac/metpetdb/wiki/ImageObject}{http://wiki.cs.rpi.edu/trac/metpetdb/wiki/ImageObject}

inorder to allow searching for images with the sample\_id or subsample\_id , a unique index is created on these columns.

\begin{tabulary}{\linewidth}{|L|L|L|}
\hline
\textbf{
Attributes
} & \textbf{
Type
} & \textbf{
Explanation
}\\\hline

image\_id
 & 
Required (PK)
 & 
Every image in the system must have an image\_id (INT)
\\\hline

checksum
 & 
Required
 & 
Checksum for the image has a 50 character limit.
\\\hline

version
 & 
Required
 & 
Version of the image.
More information \href{http://wiki.cs.rpi.edu/trac/metpetdb/wiki/Versioning}{http://wiki.cs.rpi.edu/trac/metpetdb/wiki/Versioning}
\\\hline

sample\_alias\_id
 & 
Optional (unique
index, FK)
 & 
References samples table to associate a unique sample with the image
\\\hline

subsample\_id
 & 
Optional (unique
index, FK)
 & 
References subsamples table to associate a unique subsample with the image
\\\hline

image\_format\_id
 & 
Optional (FK)
 & 
References image\_format table which specifies the format of
image (.jpg, .png etc)
\\\hline

image\_type\_id
 & 
Required (FK)
 & 
References image\_type table. List of image formats can be seen
\href{http://wiki.cs.rpi.edu/trac/metpetdb/wiki/Image\#Types}{here}
\\\hline

width
 & 
Required
 & 
Width of the image must be \textgreater{}0
\\\hline

height
 & 
Required
 & 
Height of the image must be \textgreater{}0
\\\hline

collector
 & 
Optional
 & 
The name of the collector of the image
\\\hline

description
 & 
Optional
 & 
Text description limited to 1024 characters regarding the image
\\\hline

scale
 & 
Optional
 & 
This is mm measured along the rock surface shown in the image,
corresponding to the width of the image
\\\hline

user\_id
 & 
Required (FK)
 & 
References users table for the owner of the image
\\\hline

public\_data
 & 
Required
 & 
Y/N – single character depending on whether the image is public/private.
\\\hline

checksum\_64x64
 & 
Required
 & 
Checksums are actually file addresses for each image. When an image
is loaded, it is saved on the file system in three size in three files.
Mobile one is the one shown on iphone
\\\hline

checksum\_half
 & 
Required
 & \\\hline

checksum\_mobile
 & 
Optional
 & 
Checksum for the image in the mobile app.
\\\hline

filename
 & 
Required
 & 
Name of the image file.
\\\hline
\end{tabulary}



\subsection{image\_format}
\label{Table_Description:image-format}
Stores the type of image (TIFF, GIF, JPEG, BMP, PNG,) through the image\_format\_id.

\begin{tabulary}{\linewidth}{|L|L|L|}
\hline
\textbf{
Attributes
} & \textbf{
Type
} & \textbf{
Explanation
}\\\hline

image\_format\_id
 & 
Required (PK)
 & 
Type of image
\\\hline

name
 & 
Required (unique)
 & 
Name of the image format – mime type image/x
\\\hline
\end{tabulary}



\subsection{image\_type}
\label{Table_Description:image-type}
Stores the type of image from the list of acceptable image types. The list of acceptable image types in the various image formats are specified \href{http://wiki.cs.rpi.edu/trac/metpetdb/wiki/Image}{here} (check Types)

\begin{tabulary}{\linewidth}{|L|L|L|}
\hline
\textbf{
Attributes
} & \textbf{
Type
} & \textbf{
Explanation
}\\\hline

image\_type\_id
 & 
Required (PK)
 & 
Unique id for the type of image from the list of image types
\\\hline

image\_type
 & 
Required
 & 
Name of the image type
\\\hline

abbreviation
 & 
Optional (unique)
 & 
Unique abbreviation for the image type
\\\hline

comments
 & 
Optional
 & 
Comments for the image type
\\\hline
\end{tabulary}



\subsection{regions}
\label{Table_Description:regions}
This table stores the region\_id and name where samples have been found. This is referenced by sample\_regions to correlate the region\_id and sample\_id. This is done to facilitate search of samples based on regions. More information can be found
\href{http://wiki.cs.rpi.edu/trac/metpetdb/wiki/Region}{here}

\begin{tabulary}{\linewidth}{|L|L|L|}
\hline
\textbf{
Attributes
} & \textbf{
Type
} & \textbf{
Explanation
}\\\hline

region\_id
 & 
Required (PK)
 & 
Unique id for the region from the list of regions
\\\hline

name
 & 
Required (unique)
 & 
Name of the region
\\\hline
\end{tabulary}



\subsection{sample\_regions}
\label{Table_Description:sample-regions}
This table correlates the sample\_id (for the samples found in a particular region) with region\_id from the regions table. There can be many samples in a particular region or the same sample could be found in different regions.

Sample table attributes can be found \href{http://wiki.cs.rpi.edu/trac/metpetdb/wiki/SampleObject}{here}

List of regions can be found \href{http://wiki.cs.rpi.edu/trac/metpetdb/wiki/Region}{here}

\begin{tabulary}{\linewidth}{|L|L|L|}
\hline
\textbf{
Attributes
} & \textbf{
Type
} & \textbf{
Explanation
}\\\hline

region\_id
 & 
Required (PK, FK)
 & 
Unique id for the region obtained by referencing the regions table
\\\hline

name
 & 
Required (PK, FK)
 & 
Name of the region as in the list of regions.
\\\hline
\end{tabulary}



\subsection{metamorphic\_regions}
\label{Table_Description:metamorphic-regions}
Metamorphic regions are the polygons defined by the MetPetDB team and are different from Regions.
The same Region may be named differently by different people whereas metamorphic\_regions are unique in that respect.
The id is unique to each table.
This table also stores the shape of the gis polygon as an additional column.

\begin{tabulary}{\linewidth}{|L|L|L|}
\hline
\textbf{
Attributes
} & \textbf{
Type
} & \textbf{
Explanation
}\\\hline

metamorphic\_region\_id
 & 
Required (PK)
 & 
Unique id of the metamorphic region
\\\hline

name
 & 
Required (unique)
 & 
Name of the metamorphic region
\\\hline

description
 & 
Optional
 & 
Textual description of the metamorphic region
\\\hline

shape
 & 
Optional
 & 
Column of gis polygon is added be calling the function
AddGeometrycolumn. i.e polygon of the location.
\\\hline

label\_location
 & 
Optional
 & \\\hline
\end{tabulary}



\subsection{sample\_metamorphic\_regions}
\label{Table_Description:sample-metamorphic-regions}
This table correlates the samples with the metamorphic regions by referencing the samples (for sample\_id) and metamorphic\_regions (for metamorphic\_region\_id). The 2 attributes are together unique.

\begin{tabulary}{\linewidth}{|L|L|L|}
\hline
\textbf{
Attributes
} & \textbf{
Type
} & \textbf{
Explanation
}\\\hline

sample\_id
 & 
Required (PK, FK)
 & 
Unique sample id from the samples table
\\\hline

metamorphic\_region\_id
 & 
Required (PK, FK)
 & 
Corresponding unique metamorphic region id from the
metamorphic\_regions table.
\\\hline
\end{tabulary}



\subsection{metamorphic\_grades}
\label{Table_Description:metamorphic-grades}
Metamorphic grades are one of the sample object attributes and this table stores the metamorphic\_grade\_id and name from the list of metamorphic grades \href{http://wiki.cs.rpi.edu/trac/metpetdb/wiki/MetamorphicGrade}{http://wiki.cs.rpi.edu/trac/metpetdb/wiki/MetamorphicGrade}

\begin{tabulary}{\linewidth}{|L|L|L|}
\hline
\textbf{
Attributes
} & \textbf{
Type
} & \textbf{
Explanation
}\\\hline

metamorphic\_grade\_id
 & 
Required (PK)
 & 
Unique id for the metamorphic grades from the list of metamorphic
grades
\\\hline

name
 & 
Required (unique)
 & 
Name of the metamorphic grade.
\\\hline
\end{tabulary}



\subsection{sample\_metamorphic\_grades}
\label{Table_Description:sample-metamorphic-grades}
This table correlates the samples with their metamorphic grades.

\begin{tabulary}{\linewidth}{|L|L|L|}
\hline
\textbf{
Attributes
} & \textbf{
Type
} & \textbf{
Explanation
}\\\hline

sample\_id
 & 
Required (PK, FK)
 & 
Unique sample id from the samples table
\\\hline

metamorphic\_grade\_id
 & 
Required (PK, FK)
 & 
Corresponding unique metamorphic grade id from the
metamorphic\_grades table.
\\\hline
\end{tabulary}



\subsection{georeference}
\label{Table_Description:georeference}
georeference describes detailed info (author, etc.) about a reference. But does not require that the reference be already stored in the reference table.  This way, detailed reference info can be stored in the db even if no sample references it yet.
This is important because people may upload (geo)references first and then the samples, or samples first and then georeferences.

\begin{tabulary}{\linewidth}{|L|L|L|}
\hline
\textbf{
Attributes
} & \textbf{
Type
} & \textbf{
Explanation
}\\\hline

georef\_id
 & 
Required (PK)
 & 
Unique id for the file being referred
\\\hline

title
 & 
Required
 & 
Title of the file being referred to
\\\hline

first\_author
 & 
Required
 & 
Name of the first author
\\\hline

second\_authors
 & 
Required
 & 
Name of the second author
\\\hline

journal\_name
 & 
Required
 & 
Name of the journal in which the reference appears
\\\hline

full\_text
 & 
Required
 & 
I suppose this stores the entire content of the reference?
\\\hline

reference\_number
 & 
Optional
 & \\\hline
\end{tabulary}



\subsection{sample\_georeference}
\label{Table_Description:sample-georeference}
sample\_georeference is a by product of a join between sample\_reference, reference and georeference. Whenever the georeference table is updated, this table is repopulated.. The text files are referenced via the reference\_number.


\subsection{reference}
\label{Table_Description:reference}
This table stores the reference\_id of the reference and the “name” of the file referenced.
A reference is a known unique value (also called geo-reference in outside world.)
A reference is added only when someone adds a reference for a sample.

You can have a reference even if you do not have any details about that reference in the db. You just list the info. And it is stored in reference table, and the association between reference and samples is stored in the sample\_reference table (a sample can have multiple references).

\begin{tabulary}{\linewidth}{|L|L|L|}
\hline
\textbf{
Attributes
} & \textbf{
Type
} & \textbf{
Explanation
}\\\hline

reference\_id
 & 
Required (PK)
 & 
Unique id of the reference
\\\hline

name
 & 
Required
 & 
Generally the title of the file being referenced.
\\\hline
\end{tabulary}



\subsection{sample\_reference}
\label{Table_Description:sample-reference}
This table correlates the sample\_id from the samples table with the reference\_id from the reference table which stores “name” of the referenced file.

\begin{tabulary}{\linewidth}{|L|L|L|}
\hline
\textbf{
Attributes
} & \textbf{
Type
} & \textbf{
Explanation
}\\\hline

sample\_id
 & 
Required (PK, FK)
 & 
References sample\_id from the samples table.
\\\hline

reference\_id
 & 
Required (PK, FK)
 & 
References the corresponding reference\_id from the reference table.
\\\hline
\end{tabulary}



\subsection{projects}
\label{Table_Description:projects}
Users can share private data by starting projects, inviting members etc and to organize suites of public, private and published data. This table stores information about all the projects along with version.

More information about the Projects table can be found \href{http://wiki.cs.rpi.edu/trac/metpetdb/wiki/Projects}{here}

\begin{tabulary}{\linewidth}{|L|L|L|}
\hline
\textbf{
Attributes
} & \textbf{
Type
} & \textbf{
Explanation
}\\\hline

project\_id
 & 
Required (PK)
 & 
Unique id of the project
\\\hline

version
 & 
Required
 & 
It is the number of times that specific tuple has been modified
(used for concurrency issues). You read a tuple with its version info.
When trying to save it, you check whether the version has not changed
since you read it. If not, you can write the new update. Otherwise,
the update can fail. More information \href{http://wiki.cs.rpi.edu/trac/metpetdb/wiki/Versioning}{here}
\\\hline

user\_id
 & 
Required(FK, unique)
 & 
References the users table for the owner of the
project
\\\hline

name
 & 
Required (unique)
 & 
Name of the project
\\\hline

description
 & 
Required
 & 
A textual description of the project. (upto 300 characters)
\\\hline
\end{tabulary}



\subsection{project\_members}
\label{Table_Description:project-members}
This table correlates the user\_id and the project. (i.e project\_id)
A user can be part of many projects.

\begin{tabulary}{\linewidth}{|L|L|L|}
\hline
\textbf{
Attributes
} & \textbf{
Type
} & \textbf{
Explanation
}\\\hline

project\_id
 & 
Required (PK, FK)
 & 
Referenced from the projects table.
\\\hline

user\_id
 & 
Required (PK, FK)
 & 
Corresponding user\_id referenced from the users table.
\\\hline
\end{tabulary}



\subsection{project\_invites}
\label{Table_Description:project-invites}
This table stores the invites(invite\_id, I suppose this gives the number of users associated with a project) for a particular project.

\begin{tabulary}{\linewidth}{|L|L|L|}
\hline
\textbf{
Attributes
} & \textbf{
Type
} & \textbf{
Explanation
}\\\hline

invite\_id
 & 
Required (PK)
 & 
Unique id of the invite to be a part of the project
\\\hline

project\_id
 & 
Required (FK)
 & 
Id of the project referenced from the projects table
\\\hline

user\_id
 & 
Required (FK)
 & 
ID of the project owner
\\\hline

action\_timestamp
 & 
Required
 & \\\hline

status
 & 
Optional
 & \\\hline
\end{tabulary}



\subsection{project\_samples}
\label{Table_Description:project-samples}
This table correlates the project a sample is associated with by referencing the samples (sample\_id) and the projects( project\_id) tables. (no primary key)

\begin{tabulary}{\linewidth}{|L|L|L|}
\hline
\textbf{
Attributes
} & \textbf{
Type
} & \textbf{
Explanation
}\\\hline

project\_id
 & 
Required (FK)
 & 
Referenced from the projects table
\\\hline

sample\_id
 & 
Required (FK)
 & 
Referenced from the samples table
\\\hline
\end{tabulary}



\chapter{End Notes}
\label{Notes:notes}\label{Notes::doc}\label{Notes:end-notes}
Details of the possible up gradations to the schema can be found \href{https://github.com/metpetdb/metpetdb-py/blob/master/database/MetPetDBSchemav1.0.pdf}{here}

The current \href{https://github.com/metpetdb/metpetdb/tree/master/mpdb-server/schema}{schema} may be slightly outdated and will be updated shortly.

Some of the tables not discussed in this report but are there in the schema:
Table Name: xray\_image
Probably like a subclass of images. For each image created, an xray\_image is created referencing the actual image. Not sure about what each attribute holds.

Table Name: grids, image\_on\_grid, geometry\_columns, sample\_ref\_sys (not sure about their use)

\textbf{Empty tables:}
\begin{itemize}
\item {} 
Admin\_users

\item {} 
Role\_changes

\item {} 
Users\_roles

\item {} 
Image\_comments (may be used later)

\item {} 
Image\_reference (may be used later)

\end{itemize}


\chapter{Indices and tables}
\label{index:indices-and-tables}\begin{itemize}
\item {} 
\emph{genindex}

\item {} 
\emph{modindex}

\item {} 
\emph{search}

\end{itemize}



\renewcommand{\indexname}{Index}
\printindex
\end{document}
